\chapter{Mobius transformations}In this chapter we will study a very natural class of transformations of which the inversion is a particular case. We will see connexions to complex analysis and hyperbolic geometry.%\subsection{Circles passing through a pair of symmetric points}%Let $P$ and $P'$ be a pair of points symmetric with respect to a circle $C$ (i.e. $OP\cdot OP'=R^2$, where $O$ is the center of the circle $C$ and $R$ is its radius). Suppose also that $P$, $P'$ don't lie on $C$. We claim that every circle passing through a pair of symmetric points is orthogonal to the inversion circle.%Indeed, let $S$ be a circle passing through $P$ and $P'$. Since one of the points $P$ and $P'$ is inside $C$ and the other one is outside, $S$ should intersect the circle $C$ at two points. Denote them by $A_1, A_2$. After inversion in circle $C$ the points $A_1, A_2$ will stay where they are (all points on the inversion circle are fixed by the inversion) and point $P$ will go to point $P'$ (by assumption that they are symmetric with respect to $C$). Hence the circle $S$, being the only circle passing through the three points $P, A_1, A_2$ gets inverted to the only circle passing through $P', A_1$ and $A_2$. Hence the circle $S$ gets inverted to itself. The angle formed by $C$ and $S$ at point $A_1$ gets mapped by the inversion to the complimentary angle at this point. Since the magnitudes of angles are preserved by inversion, each of these angles must be half of $180 \degree$, i.e. a right angle.%We've established that every circle passing through a pair of symmetric points is orthogonal to the inversion circle. Now can also prove the converse: if a circle $S$ passes through point $P$ and is orthogonal to the circle $C$, then it passes also through the point $P'$, the image of $P$ after the inversion in $C$. Indeed, the ray joining the center $O$ of the circle $C$ with the point $P$ intersects $S$ at one other point $Q$. The ray from $O$ to $P$ maps to itself under the inversion in $C$ and so does the circle $S$ (the points of intersection of $C$ and $S$ map to themselves, since they belong to $C$, and there is only one circle passing through two given points on $C$ and orthogonal to it; hence it must map to itself after inversion). This proves that the pair of points of intersection of $S$ with the ray maps to itself. Note however that $P$ can't be fixed by inversion (it doesn't belong to $C$) and hence must be sent to $Q$. This proves that $P'$, the image of $P$ under inversion, is equal to $Q$ and hence belongs to $S$.%Now consider the following picture: let the pair of points $P$, $P'$ be given. Associated to this pair are two pencils of circles: first pencil consists of all circles $S$ passing through $P$ and $P'$ and the second consists of all circles $C$ with the property that the the points $P$ and $P'$ are symmetric in $C$. Then it follows from what we've proved above that every circle from the first pencil is orthogonal toe every circle from the secon one. We will consider this picture once more after we will study Mobius transformations.\subsection{Introduction}Our discussion of inversion in the plane was rather unsatisfying from the point of view of Klein's definition of what geometry is. Of course we introduced a very beautiful class of transformations - inversions send lines or circles to lines or circles and preserves angles up to a sign. But if we want to study a geometry which is preserved by inversions, we should embed them into a group. For instance, we know almost everything about a single inversion; but what about a composition of two of them? It is also a wonderful transformation - it preserves angles and sends lines or circles to lines or circles. And what about a composition of three? Is it an inversion in some circle?One approach to defining inversive" geometry - that is a geometry where the group of motions contains all inversions - is to say that a transformation is in the group if it is a composition of a finite number of inversions.Such an approach is somewhat too abstract - it seems quite complicated, given a transformation, to decide whether it is a composition of inversions or not. Of course such a transformation should send lines or circles to lines or circles and preserve angles (maybe up to a sign), but the converse is not clear.Another approach is to define our group of motions as the group of all transformations that send lines or circles to lines or circles and preserve angles. This is clearly a group that contains all inversions. Now the problem is quite opposite to the problem in the previous approach. We would like to find a simple sub-collection which generates this group, but it is not at all clear how to go about this hunt. If we have such a collection it is easy to verify that all our transformations satisfy some property (e.g. preserving angles). We only need to check that the transformations from the collection satisfy this property and that this property survives composition.In fact both these approaches lead to the same answer. Moreover, in the plane this answer can be made very concrete. We think of the plane $R^2$ as the complex line $C$.\subsection{Fractional linear transformations in the extended complex line}We will start our journey into the world of Mobius geometry by a study of a very concrete group of transformations of the extended complex line that preserve angles and generalized circles. The extended complex line is $C$ with a point at infinity $\infty$ adjoined. A generalized circle is a circle which may or may not intersect this point at infinity.Recall first how the arithmetic of complex numbers works:1. To add two complex numbers $z_1,$and$z_2$, we just add the corresponding vectors (each vector joining 0 and $z_i$).2. To multiply two complex numbers $z_1,$and$z_2$, we should multiply their magnitudes (i.e. the lengths of the corresponding vectors) and add their arguments (i.e. the angles the vectors make with the ray of positive real numbers): $\left|z_1 z_2\right|=\left|z_1\right|\left|z_2\right|$, $arg(z_1 z_2)=arg(z_1)+arg(z_2)$.3. The conjugate of the number $z$, denoted $\bar{z}$, is the reflection of $z$ in the real axis.4. The inverse of a non-zero complex number $z$ is the unique number $w=\frac{1}{z}$ with the property $z w=1$: $\left|\frac{1}{z}\right|=\frac{1}{\left|z\right|}$,$arg(1/z)=-arg(z)$. It is equal to $\frac{\bar{z}}{\left|z\right|^2}$.The last identity has the following geometrical meaning: the transformation $z\rightarrow 1/z$ is the composition of reflection in the real line, $z$goes to$\bar{z}$, with inversion in the unit circle $z\rightarrow z/|z|^2$. $z/|z|^2$ is the point on the line connecting the origin to $z$ which distance from the origin is equal to $1/|z|$. Since reflection and inversion both flip the signs of angles and send generalized circles to generalized circles, their composition preserves angles and sends generalized circles to generalized circles.Here are some other examples of transformations that have the same property: they preserve angles and send generalized circles to generalized circles.\begin{enumerate}\item The translations $\z\rightarrow \z+c$. Translations are Euclidean transformations: they preserve everything: distances, angles, shapes et c.\item The dilations $\z\rightarrow \lambda \z$ with $\lambda$ real and positive.\item The rotations $\z\rightarrow \lambda \z$ with $\lambda$ complex number of magnitude 1 (this is the rotation about the origin by angle equal to argument of $\lambda$).\item The transformation $\z\rightarrow \lambda \z$ with $\lambda\neq 0$ is a composition of dilation by $|\lambda|$ and rotation about $0$ by angle $\arg(z)$.\item The inversion" $\z\rightarrow \frac{1}{\z}$. As we explained above, this is a composition of reflection in the real line with inversion in the unit circle.\end{enumerate}By composing maps from five listed above, we always get mappings of the form $z\rightarrow \frac{a z+b}{c z + d}$ with $a,b,c,d \in \mathbb{C}, a d - b c \neq 0$. In fact any transformation of this kind can be obtained by a composition of mappings from the examples above. Indeed, if $c\neq 0$, then we can divide $a \z+b$ by $cz +d$ with remainder: $\frac{a \z+b}{c \z + d}=\frac{a}{c}+\frac{b c - a d}{c}\cdot\frac{1}{c \z+d}$. So the transformation $z\rightarrow \frac{a z+b}{c z + d}$ is a composition of multiplication by $c$, translation by $d$, inversion, multiplication by $\frac{b c - a d}{c}$ and translation by $\frac{a}{c}$. We see from this formula the need for the condition $a d - b c \neq 0$. If it were zero, our transformation would send any point to the constant point $\frac{a}{c}$. In the case $c=0$, it's even simpler: $\frac{a \z+b}{c \z + d}=\frac{a}{d}\cdot \z + \frac{b}{d}$.A transformation given by $z\rightarrow \frac{a \z+b}{c \z + d}$ with $a,b,c,d \in \mathbb{C}, a d - b c \neq 0$ is called a fractional linear transformation, or a Mobius transformation. We have proved that every Mobius transformation preserves generalized circles and angles between them (with orientation).A simple observation (that requires some checking) is that fractional linear transformations form a group under composition. That is the composition of two fractional linear transformations is fractional linear, and every transformation $\z\rightarrow \frac{a \z+b}{c \z + d}$ with $a,b,c,d \in \mathbb{C}, a d - b c \neq 0$ has an inverse $\z\rightarrow \frac{d \z - b}{-c \z + a}$. The composition of these transformations in either order is the identity transformation.Note that if $z_1,z_2,z_3 \in \mathbb{C}$ are distinct, then the fractional linear transformation $T^{z_1,z_2,z_3}_{0,1,\infty}: \z \rightarrow \frac{z_2-z_3}{z_2-z_1}\cdot\frac{\z-z_1}{\z-z_3}$ sends $z_1$ to $0$, $z_2$ to $1$ and $z_3$ to $\infty$. So if $u_1,u_2,u_3 \in \mathbb{C}$ is another triple of distinct complex numbers, then there is a fractional linear transformation $T^{z_1,z_2,z_3}_{u_1,u_2,u_3}$ mapping $z_1$ to $u_1$, $z_2$ to $u_2$ and $z_3$ to $u_3$. This transformation is given by the composition $\left(T^{u_1 u_2 u_3}_{0,1,\infty}\right)^{-1}\circ T^{z_1 z_2 z_3}_{0,1,\infty}$.The next theorem gives an important geometric characterization of fractional linear transformations.\begin{theorem}Let $f$ be a one-to-one transformation of the extended complex line that sends generalized circles to generalized circles and preserves angles. Then $f$ is a fractional linear transformation.\end{theorem}\begin{proof}Let $z_1,z_2,z_3$ be the images of points $0,1$ and $\infty$ under the transformation $f$. There exists a fractional linear transformation $T^{z_1,z_2,z_3}_{0,1,\infty}$ that sends $z_1,z_2,z_3$ back to $0,1,\infty$. The composition of $f$ and $T^{z_1,z_2,z_3}_{0,1,\infty}$ is a transformation that fixes $0,1$ and $\infty$, preserves angles and sends generalized circles to generalized circles. If we prove that this composition is the identity, it will follow that $f$ is the inverse of $T^{z_1,z_2,z_3}_{0,1,\infty}$, so it must be fractional linear.Now let us prove that if $g$ is a transformation that fixes $0,1$ and $\infty$, preserves angles and sends generalized circles to generalized circles, then it is the identity transformation.The real line is the only generalized circle passing through the points $0,1$ and $\infty$, so it must be sent to itself by $g$ (that is not to say that each point is fixed, only that the set is its own image).Let $z$ be any point not on the real line. Connect $z$ to point $0$ by a line $l_0$. This line is the only generalized circle that passes through the points $z$, $0$ and $\infty$. Hence it must be mapped to the only generalized circle that passes through $g(z)$,$0$ and $\infty$. Since generalized circles passing through $\infty$ are lines, the image of this line is a line connecting $g(z)$ with $0$.Similarly the image of the line $l_1$ connecting the point $z$ to the point $1$ is the line connecting the point $g(z)$ to the point $1$.Now we use that $g$ preserves angles: the angle formed by $l_0$ and the real line must be equal to the angle formed by line $g(l_0)$ and the real line. Hence $g(l_0)$ must coincide with $l_0$. Similarly the image $g(l_1)$ must coincide with the line $l_1$. Thus the only finite intersection point $z$ of lines $l_0$ and $l_1$ must get mapped to the only intersection point of lines $g(l_0)$ and $g(l_1)$, namely itself.Thus $g$ fixes every point $z$ not lying on the real line.It is easy to see from this that in fact $g$ fixes all points of the extended complex line, i.e. it is the identity mapping (e.g. we can repeat the argument above with a line parallel to the real line).\end{proof}Remark: if we want to study the transformations that preserve angles but reverse their orientations (while still sending generalized circles to generalized circles), we can proceed as follows. First apply the reflection $z\rightarrow \bar{z}$, and then apply our transformation $g$. The composite transformation $z\rightarrow g(\bar{z})$ preserves angles and is thus of the form $\z\rightarrow \frac{a \z+b}{c \z + d}$ with $a,b,c,d \in \mathbb{C}, a d - b c \neq 0$. Hence the original transformation $g$ is of the form $z\rightarrow \frac{a \bar{z}+b}{c \bar{z} + d}$.\subsection{Hyperbolic geometry}The fractional linear transformations form a rather rich family of transformations: for instance every three points can be mapped to three other given points by such a transformation. This is much more freedom than what we had for the Euclidean motions, where all we can do is map any one point to any other point.Even if we restrict our view to transformations that map some given circle to itself, we still get a richer family. To see this, we can think of the transformations that send the real line to itself. If the numbers $a,b,c,d$ in the transformation $z\rightarrow (az+b)/(cz+d)$ are all real, then any real number $z$ stays on the real line. So we get a three-dimensional family of transformations preserving the real line (three-, not four-, because we can always scale the numbers $a,b,c,d$ by a constant and get the same transformation).In fact this family of transformations can be used as a basis for a geometry which is in many respects similar to the Euclidean geometry. More precisely, we can study the geometry on the upper half plane $Im(z)>0$ where the group of motions is the group of transformations $z\rightarrow (az+b)/(cz+d)$ with real $a,b,c,d$ and $ad-bc>0$ (the second condition is necessary if we want our transformations to map the upper half-plane to the upper half-plane, rather than the lower one).In this geometry one can define the lines" to be the generalized semicircles which are orthogonal to the real axis. We already know from the properties of fractional linear transformations that these lines" get mapped to lines". Also, it's easy to see that through any two points passes a unique line": the semicircle passing through points $P$ and $Q$ and orthogonal to the real line should have centre on the intersection of perpendicular bisector to the segment $PQ$ and the real line.We define angles in this geometry to be the usual Euclidean angles. We already know that these are preserved by the transformations in the group of motions we are talking about. In fact one can go pretty far along this road and define many notions we are familiar with from Euclidean geometry, like distance, circles, areas et c., and prove analogues of familiar theorems about planar geometry.This geometry is called hyperbolic geometry. In hyperbolic geometry all axioms that were used by Euclid to define the usual planar geometry hold, besides one - the axiom that states that for every line $l$ and any point $P$ not on the line there exists exactly one line which is parallel to $l$ and passing through $P$. In hyperbolic geometry the corresponding axiom should be changed to ``for every line $l$ and point $P$ not on this line there exist two distinct lines $l'$ and $l''$ passing through point $P$ and parallel to $l$ such that any line between them is also parallel to $l$".We are not going to study hyperbolic geometry in detail. We will instead study spherical geometry, which has a very simple model but many of the features of hyperbolic geometry that differentiate it from the Euclidean.\section{How to construct things using ruler and compass}One of the questions in which the ancients were interested was which constructions can be performed using only a ruler and a compass. What we mean by a ruler is just a tool that enables one to construct the straight line through two given points. A compass is a tool that constructs a circle given its centre a point on it. These are abstractions of the physical objects ruler" and compass," so they can do nothing else.Here is an example of a ruler and compass construction:Given two points $A$ and $B$ we can construct the midpoint of the segment $AB$ by the following procedure. Connect the points $A$ and $B$ by a segment. Construct circle $C_A(B)$ with center at $A$ containing $B$. Construct circle $C_B(A)$ with center at $B$ containing $A$. Connect the two intersection points of these two circles by a straight line. The intersection of this line and the segment $AB$ is the midpoint of $AB$, because the picture is symmetric with respect to the reflection that interchanges $A$ and $B$.Note that in this construction we found not only the midpoint of $AB$, but also the perpendicular bisector of the segment $AB$.\begin{figure}[h]\centering\begin{asy}import geometry;size(4cm);point A=(0,0), B=(1,0);dot("$A$",A,W); dot("$B$",B);draw(segment(A,B));circle CA=circle(A,1), CB=circle(B,1);draw(CA); draw(CB);point i1=intersectionpoints(CA,CB)[0], i2=intersectionpoints(CA,CB)[1];line L = line(i1,i2); draw(L);markrightangle(A,intersectionpoint(line(A,B),L),i2);\end{asy}\label{constructing the midpoint}\end{figure}We can use the constructions we know to build up gradually more sophisticated ones. For example, given three non-collinear points $A,B,$and$C$, we can construct the circle passing through them. Indeed, its center is located at the intersection point of perpendicular bisectors of $AB$ and $BC$. Since we know how to construct these, we can construct the center of the circle we need, and then construct the circle itself using the compass.Once we know that we can construct the circumscribed circle of a triangle, it's natural to ask about the inscribed one. Since the centre of the inscribed circle is the point of intersection of angle bisectors, all we have to do is to learn how to construct a bisector of a given angle $\angle ABC$.This is easy: first we construct a circle with center at $B$ and any radius. Let $A'$ and $C'$ be its points of intersection with lines $AB$ and $BC$ respectively. Now construct two circles of any equal radius with centres at $A'$ and $C'$. These circles intersect twice and connecting the intersections gives the bisector of $\angle ABC$.\begin{figure)[h]\centering\begin{asy}import geometry;size(8cm);point B=(0,0), A=(3,1), C=(1,3);dot("$A$",A); dot("$B$",B,S); dot("$C$",C);draw(segment(B,A)); draw(segment(B,C));circle circ = circle(B,1);point aprime = intersectionpoints(segment(B,A),circ)[0];dot("$A\prime$",aprime);point cprime = intersectionpoints(segment(B,C),circ)[0];dot("$C\prime$",cprime);circle acirc=circle(aprime, .5), ccirc=circle(cprime,.5);draw(circ); draw(acirc); draw(ccirc);draw(line(B,(1,1)));\end{asy}\label{constructing the bisector}\end{figure}Our next task will be to divide a given segment into $n$ equal parts (where $n$ is some given number). If we knew how to pass a line parallel to a given one through a given point, the following procedure would work. Say that you wish to divide segment $AB$. Then pass any line through point $A$ which does not contain the point $B$. On this line mark $n$ segments $AA_1$, $A_1A_2$,...,$A_{n-1}A_n$ of equal length (this can be done easily using a compass). Now connect $A_n$ to $B$ by a straight line and pass lines parallel to it through all other points $A_i$. Let $B_i$ be the intersection points of these lines with segment $AB$. Then points $B_1,\ldots,B_n=B$ subdivide the segment $AB$ into $n$ equal parts.\begin{figure][h]\centering\begin{asy}import geometry;size(8cm);point A=(0,0), B=(1,0), a1=(.5,.5), a2=(1,1), a3=(1.5,1.5);dot("$A$",A,W); dot("$B$",B); draw(segment(A,B));dot("$A_1$",a1,W);dot("$A_2$",a2,W);dot("$A_3$",a3,W);draw(line(A,a3));line l3=line(a3,B), l2=parallel(a2,l3), l1=parallel(a1,l3);draw(l3); draw(l2); draw(l1);\end{asy}\label{dividing a segment into equal parts}\end{figure}The only component we are still missing is the construction of a line parallel to a given line $l$, passing through a given point $A$ not on $l$. We can construct such a line as follows. Choose any point $B$ on the line $l$. Let point $C$ be the other point of intersection of a circle centred at $A$ with radius $AB$ and the line $l$. There are two angle bisectors between lines $AB$ and $AC$. One is perpendicular to $l$, the other is parallel to the line $l$.Another natural question to ask is whether we can subdivide a given angle into $n$ equal parts. The answer to this question is much less obvious than what we have yet done, and we shall devote the next chapter to answering it. It turns out that most angles cannot be trisected, for example. Questions like whether some given angle (say 90 degrees) can be subdivided into 17 equal parts are rather deep and required the genius of a young Gauss to answer.One of the most complicated constructions known to ancients was the construction of a circle tangent to three given ones. This problem is known as Appolonius' problem.We will study this problem only in the case that not all three circles intersect each other.To be covered:\begin{itemize}\item Appolonius problem - constructing a circle tangent to three given circles.\item Using inversion to simplify Appolonius problem.\end{itemize}