\chapter{Conics in projective geometry}

Recall that in projective geometry we deal with notions which are invariant under projections. For instance the circle is not a meaningful figure - after projection from one plane to another it can be transformed into an ellipse, a parabola or a hyperbola. Indeed, consider the cone over this circle with vertex at the center of projections. The image of the circle under projection to a plane $\pi'$ is the intersection of this cone with the plane $\pi'$. The figures that we can get by intersecting a cone with a plane are called conic sections. One can prove that they are in fact just ellipses, parabolas and hyperbolas.

So instead of the many different notions of Euclidean geometry, there is only one figure, a conic section, in the projective world. This observation can be often used when we are solving problems about conics. If we see that the problem can be formulated using notions that are invariant under projective transformations, then we can choose a transformation that makes our problem simpler.

Our first goal is to learn one of the quantities that are preserved by projective transformations: the cross ratio of points on a conic.

\section{Cross-ratio of four points on a conic}

In this section we define the notion of cross ratio of four points on a conic and prove its projective invariance.

Consider a conic $E$ and four points $A,B,C,$ and $D$ on it. Choose any other point $X$ on the conic $E$ and any line $l$ in the plane not passing through $X$. The lines $XA$,$XB$,$XC$ and $XD$ intersect the line $l$ in points $A'$,$B'$,$C',$ and $D'$. We define the cross ratio of points $A,B,C,$ and $D$ on the conic $E$ to be equal to the cross-ratio of points $A',B',C',$ and $D'$ on the line $l$.

This definition looks rather dubious: what if we choose a different point $X'$ on $E$ or a different line $l'$? Won't we get a different value? No.

We shall prove now that this cross ratio is independent of the choices of $X$ and $l$ and is therefore well-defined.

First note that the definition doesn't depend on the choice of line $l$: if $l'$ is a different line and $A'',B'',C'',D''$ are the points of its intersection with the lines $XA,XB,XC,XD$, then the cross-ratios $(A',B',C',D')$ and $(A'',B'',C'',D'')$ are equal, as was proved in the preceding chapter.

\includegraphics[height=60mm]{./chapter10/conicprj.png}

Now we want to prove that if $X_1$ and $X_2$ are different points on the ellipse and $A_1,B_1,C_1,D_1$ and $A_2,B_2,C_2,D_2$ are the points of intersection of line $l$ with the lines $X_1A,X_1B,X_1C,X_1D$ and $X_2A,X_2B,X_2C,X_2D$ respectively, then the cross ratios $(A_1,B_1,C_1,D_1)$ and $(A_2,B_2,C_2,D_2)$ are equal. Note that the question whether the cross ratios thus constructed are equal is invariant under projective transformations. Thus we can apply any projective transformation to our picture and prove the claim for the picture we get after the application of this transformation. A natural transformation in this case would be one that maps the conic to a circle.

\includegraphics[height=60mm]{./chapter10/conicprj2.png}

So now we are left with the same question as before, but instead of an arbitrary conic we have a circle. For the circle, the question is rather simple: as we have seen the cross ratio $(A_1,B_1,C_1,D_1)$ is equal to $\frac{\sin\angle A_1X_1 C_1}{\sin \angle B_1X_1C_1}\div\frac{\sin\angle A_1X_1D_1}{\sin \angle B_1X_1D_1}$, which is the same as $\frac{\sin\angle AX_1 C}{\sin \angle BX_1C}\div\frac{\sin\angle AX_1D}{\sin \angle BX_1D}$. Similarly the cross-ratio $(A_1,B_1,C_1,D_1)$ is equal to $\frac{\sin\angle AX_2 C}{\sin \angle BX_2C}\div\frac{\sin\angle AX_2D}{\sin \angle BX_2D}$. Now the equality of the two cross ratios follows from the equalities of angles $\angle AX_1C=\angle AX_2C$,$\angle BX_1C=\angle BX_2C$, $\angle AX_1D=\angle AX_2D$ and $\angle BX_1D=\angle BX_2D$, as these pairs are each subtended by the same arc.

\subsection{A different definition of cross-ratio}

We can define the cross ratio of four points on a quadric in a quite different way: choose a line $l_X$ tangent to the quadric $E$ at point $X$ on it. Let $l_A$, $l_B$, $l_C$, $l_D$ denote the lines tangent to the quadric at points $A$,$B$,$C$, and $D$ respectively. Let $A',B',C',$ and $D'$ denote the points of intersection of lines $l_A,l_B,l_C,$ and $l_D$ with the line $l_X$. Then we can define the cross-ratio of points $A,B,C,D$ as the cross-ration of points $A',B',C',D'$ on the line $l_X$. We claim that this definition gives the same value as the previous one.

The proof of this claim is similar to what we did before: first we note that the question is invariant under projective transformations. Then we apply a transformation to make the conic a circle. And finally for the circle we use some angle chasing.

Let $E$ be a circle with center $O$ and $A,X$ be two points on it. Let $l_X$, $l_A$ be the tangent lines to the circle $E$ at the points $X$ and $A$. Let $A'$ be the intersection point of these tangent lines. Then the angle $\angle A'OX$ is equal to one half of the angle $\angle AOX$, since the two triangles $OAA'$ and $OXA'$ are congruent.

Thus we find that in our situation angles $\angle A'OC'$,$\angle A'OD'$,$\angle B'OC'$,$\angle B'OD'$ are equal to $\frac{1}{2}\angle AOC$,$\frac{1}{2}\angle AOD$,$\frac{1}{2}\angle BOC$ and $\frac{1}{2}\angle BOD$ respectively (e.g. $\angle A'OC'=\angle A'OX - \angle C'OX=\frac{1}{2} \angle AOX - \frac{1}{2} \angle COX= \frac{1}{2} \angle AOC$). Thus the double ratio $(A',B',C',D')$ is equal to $\frac{\sin\angle A'O C'}{\sin \angle B'OC'}\div\frac{\sin\angle A'OD'}{\sin \angle B'OD'}$ which is the same as $\frac{\sin\frac{1}{2}\angle AO C}{\sin \frac{1}{2}\angle BOC}\div\frac{\sin\frac{1}{2}\angle AOD}{\sin \frac{1}{2}\angle BOD}$. Now this quantity is equal to the cross-ratio we defined previously, since for any choice of point $X$ on the circle $E$ we have $\angle AXC=\frac{1}{2} \angle AOC$ and similarly for the other relevant angles.

\section{Projective maps from a line to itself}

As we have seen a projective mapping of a space to itself preserves the cross ratio of any four collinear points. This prompts the following definition: a mapping $f:l\rightarrow l'$ from a projective line $l$ to a projective line $l'$ is projective if it preserves cross-ratios: for any four points $A,B,C,D$ on $l$ we have $(A,B,C,D)=(f(A),f(B),f(C),f(D))$.

Notice that this defines what a projective transformation of a line to another line is for any two lines - they don't have to be in the same plane, nor even in the same projective space!

Let us give two examples of projective transformations in the case the two lines $l$ and $l'$ do belong to the same projective plane.

In the first example let $E$ be a point in the plane not lying on either of the two lines. Consider the transformation $f$ sending any point $A\in l$ to that point $f(A)\in l'$ which is the intersection of line $AE$ and $l'$. This is called the central projection from line $l$ to line $l'$ with centre $E$. To show it is in fact a projective transformation, we need only show that if $A,B,C,D$ are four points on the line $l$ and the lines $EA,EB,EC,ED$ intersect the line $l'$ at points $f(A),f(B),f(C),f(D)$, then $(A,B,C,D)=(f(A),f(B),f(C),f(D))$. But this is just the cross ratio of lines $EA,EB,EC,ED$ computed in two different ways - intersecting them with line $l$ and then intersecting them with line $l'$.

Note that in this example the point of intersection of lines $l$ and $l'$ gets mapped to itself.

Another example of a projective transformation is as follows: suppose that a conic $E$ is tangent to both lines $l$ and $l'$.Then we define the image of point $A\in l$ as the intersection point of line $l'$ with the tangent line to the conic $E$ from point $A$ which is different from $l$.

The fact that such transformation is projective follows from the fact that the definition of cross ratio of four points on a quadric using tangent lines is well-defined.

Indeed, if lines $Af(A),Bf(B),Cf(C),Df(C)$ are tangent to the conic $E$ at points $A',B',C',D'$, then the cross ratios $(A,B,C,D)$ and $(f(A),f(B),f(C),f(D))$ are both equal to the cross ratio of the points $A',B',C',D'$ on the conic $E$.

In fact for two lines $l$ and $l'$ in the same plane every projective transformation is of one of these two kinds. We will prove this statement in full generality later, after we study projective duality. For now let's prove only the following statement:

\begin{theorem}
Let $l$ and $l'$ be two lines in projective plane and let $f:l\rightarrow l'$ be a projective transformation from $l$ to $l'$ which maps the point of intersection of $l$ with $l'$ to itself. Then there exists a point $E$ in the plane so that the transformation $f$ is just the central projection of $l$ onto $l'$ with center $E$.
\end{theorem}

\begin{proof}
A useful rule to remember, which we will use in this proof, is this. If three different points, $A,B,$ and $C$ on a line $l$ are known, and the cross ratio of these three points with a fourth point $X$ is known, then that fourth point X is also known.
Let $A,B$ be any two points on line $l$ which are distinct from each other and from the point of intersection $C$. Let $E$ be the point of intersection of lines $Af(A)$ and $Bf(B)$. The central projection with center $E$ from $l$ to $l'$ sends $C$ to itself and $A$ and $B$ to $f(A)$ and $f(B)$. Suppose now $X\in l$ is any point on $l$ and $X'$ is the image of $X$ under central projection with center $E$ from $l$ to $l'$. Then we know that $(O,A,B,X)$ is equal to $(O,f(A),f(B),f(X))$ because $f$ is projective and we also know it is equal to $(O,f(A),f(B),X')$, because the central projection with center $E$ is projective. Since $(O,f(A),f(B),f(X))=(O,f(A),f(B),X')$, we see that $X'=f(X)$, i.e. $f$ coincides with the central projection with center at $E$.
\end{proof}


\section{Pascal's theorem}

Merely having a well-defined notion of cross-ratio of four points on a conic proves the following statement:

Let $l_1$ and $l_2$ be any two lines in the plane and $\mathfrak{E$ any conic. Suppose that $O_1$ and $O_2$ are two distinct points on the conic $\mathfrak{E$. Define the mapping $f$ from $l_1$ to $l_2$ as following: first do a projection from line $l_1$ to the conic $E$ with center at $O_1$ and then project from the conic to line $l_2$ with center $O_2$. Then the mapping $f$ is projective.

Indeed, if $A,B,C,D$ are four points on the line $l_1$ and $A',B',C',D'$ are the points of intersection of lines $O_1A,O_1B,O_1C,O_1D$ with the conic $\mathfrak{E}$, then the cross ratio $(A',B',C',D')$ is equal both to $(A,B,C,D)$ and to $(f(A),f(B),f(C),f(D))$.

Now suppose that the lines $l_1$ and $l_2$ intersect at a point $O$ on the conic $\mathfrak{E}$. Then we see that $f$ sends the point $O$ to itself. But then the theorem from the previous section tells us that there is some point $E$ in the plane such that $f$ is just the central projection from $l_1$ to $l_2$ with center $E$. Let's see whether we can find this point $E$.

Let $P_1$ denote the point of intersection of $l_1$ with $\mathfrak{E}$ and $P_2$ be the point of intersection of $l_2$ with $\mathfrak{E}$. The point $P_1$ gets mapped by the first projection to itself, and then by the second projection to the point of intersection of $l_2$ with $O_2P$. Thus the point $f(O_1)$ lies on the line $O_2P_1$.

Similarly the point of intersection of line $l_1$ with $O_1P_2$ gets mapped by the first projection to point $P_2$ and then point $P_2$ gets mapped to itself by the second projection. Thus the point of intersection of the line $O_1P_2$ with the line $l_1$ gets mapped to point $P_2$.

Thus the center of projection $E$ must be the point of intersection of $O_2 P_1$ and $O_1 P_2$.

Now let $P$ be any point on conic $\mathfrak{E}$, and let $E_1$ and $E_2$ be the intersection points of lines $PO_1$ and $PO_2$ with $l_1$ and $l_2$ respectively. Then $E_2=f(E_1)$ and hence $E_1,E_2$ and $E$ are collinear!

Let's summarize what we proved. Let $\mathfrak{E}$ be any ellipse and let $OP_1O_2PO_1P_2$ be any hexagon inscribed in it. Let $E$,$E_1$ and $E_2$ be the points of intersection of pairs of opposite sides of this hexagon, namely $E_1=OP_1\cap PO_1$, $E_2=OP_2\cap PO_2$ and $E=O_1P_2 \cap P_1O_2$. Then the points $E,E_1,E_2$ are collinear!

\includegraphics[height=70mm]{./chapter10/Pascal.pdf}

Thanks to Chang Yu and Siming Zhao for illustration.

This theorem was discovered by Blaise Pascal when he was sixteen years old. He called this theorem ``The Mystic Hexagram," and wrote a treatise on conic sections wherein this theorem and its numerous corollaries were the leit-motif.

We can prove a much older theorem, due to Pappus, along the same lines:

\begin{theorem}
Let $\mathfrak{L}_O$ and $\mathfrak{L}_P$ be two lines. Let $O_1,O_2,O_3$ be three points on line $\mathfrak{L}_O$ and let $P_1,P_2,P_3$ be three points on line $\mathfrak{L}_P$. Let $E_1$ be the point of intersection of lines $O_2P_3$ and $O_3P_2$. Similarly let $E_2=O_1P_3\cap O_3P_1$ and $E_3=O_1P_2 \cap O_2P_1$. Then the three points $E_1,E_2,E_3$ are collinear.
\end{theorem}

\includegraphics[height=70mm]{./chapter10/Pappus.png}

Thanks to Chenchen Li for the illustration

\begin{proof}
Consider the mapping $f:O_1P_2\rightarrow O_1P_3$ which is the composition of the central projection from $O_1P_2$ to $\mathfrak{L}_P$ with center at $O_2$ and the central projection from $\mathfrak{L}_P$ to $O_1P_3$ with center $O_3$. This mapping is clearly projective transformation. It maps the point $O_1$ to itself (the first central projection maps the point $O_1$ to the intersection point of $\mathfrak{L}_O$ and $\mathfrak{L}_P$, while the second central projection maps this intersection point back to $O_1$). Thus it must itself be a central projection with some center. Let us locate this center.

Consider point $P_2$. The first central projection maps it to itself. Thus the point $f(P_2)$ lies on the line $O_3P_2$.

Consider now the point $P_3$. Its preimage under the second central projection is itself, hence its preimage under $f$ lies on the line $O_2P_3$.

Thus the center of the central projection $f$ is the point $E_1$ - the intersection of lines $O_3P_2$ and $O_2P_3$.

Now let's look at point $E_3$. Its image under the first central projection is the point $P_1$ and the image of $P_1$ under the second central projection is $E_2$. Thus $f(E_3)=E_2$. Since $f$ is the central projection with center $E_1$, the points $E_1,E_2,E_3$ must be collinear.

\end{proof}

\section{Conics as rational curves}

In this section we will show that conic sections have a very useful property: they admit a parametrization by rational functions of one parameter. This property can be used in quite diverse applications: from explaining why cross-ratio can be defined for points on a conic section, to finding all Pythagorean triples, to the integration of functions that involve square roots.

\subsection{Rational parametrization}

The parametrization we are looking for is just a rational map from a line to a conic, which is one-to-one and onto and such that the inverse map is also a rational map. A curve that admits such parametrization is usually called a rational curve. Since rational curves in general are not the subject of this lecture, we won't make our definition completely precise, but instead just study the rational parametrization of a conic.

Let $O$ be a point on a conic $E$. Let $l$ be any line not passing through point $O$. For any point $A$ on the line $l$ the line $OA$ intersects the conic $E$ at two points - $O$ and some other point. To see this, we can restrict the equation of a conic, which is a quadratic equation, to the line $OA$. What we get is a quadratic equation that has one root corresponding to point $O$. Hence it has exactly one other root. (If the line $OA$ is tangent to the conic the quadratic equation will have a double root at point $O$.)

Our parametrization $\phi:l\rightarrow E$ takes the point $A\in l$ to the other point of intersection of line $OA$ with the conic $E$.

The inverse map $\psi:E\rightarrow l$ is obvious: it is the central projection from the conic $E$ to line $l$ with center $O$.

Now we claimed that both these maps are in fact rational maps. The reason it is so is that to find the point of intersection of a line with a quadric, which is different from the point which is already given to us, we don't in fact have to solve any quadratic equations. Instead we can just use Vieta's formulas.

Let us do an example to make this idea both clear and precise.

Let $E$ be the circle $x^2+y^2=1$, point $O$ be the point $(0,1)$ and the line $l$ to be the line $y=0$.

We can describe a point on a line $l$ with just one coordinate, say the $x$-coordinate of the corresponding point. So the point $A=(0,t)\in l$ will have coordinate $t$.

Now the line $OA$ has the equation $t(1-y)=x$ (check that it passes through both points $(0,1)$ and $(0,t)$). To intersect it with the circle $E$ we have to solve the system of equations
\begin{align*}
x^2+y^2=1\\
t(1-y)=x
\end{align*}

By plugging the second equation to the first we do the operation we described before: restrict the equation of a quadric $E$ to the line $OA$. What we get is the equation $t^2(1-y)^2+y^2=1$ or $(t^2+1)y^2-2t^2 y +t^2-1=0$. It seems that it is a quadratic equation in $y$ and in order to solve it we need to extract square roots. The beautiful idea here is that we already know one of the solutions of this equation: point $(0,1)$ belongs to the intersection of $OA$ and $E$, and thus $y=1$ is a solution. To find the other solution all we have to do is use Vieta's formula. It tells us that the product of the two roots is equal to $\frac{t^2-1}{t^2+1}$. Since one of them is $1$, the other one is $\frac{t^2-1}{t^2+1}$. From $x=t(1-y)$ we find that corresponding to $y=\frac{t^2-1}{t^2+1}$ we have $x=\frac{2t}{t^2+1}$.

Thus we have that the parametrization of the circle $E$ we were after is $\phi(t)=(\frac{2t}{t^2+1},\frac{t^2-1}{t^2+1})$.

The inverse map $\psi$ can be found by looking at triangles with vertices at $O$,$(0,0)$ and $(0,t)$ and with vertices $O$,$(0,y)$,and $(x,y)$ (where $(x,y)\in E$ is the point on $E$ that gets mapped to the point with coordinate $t$ by $\psi$). Since these triangles are similar, we have $\frac{x}{t}=\frac{1-y}{1}$, or $t=\frac{x}{1-y}$. Thus $\psi(x,y)=\frac{x}{1-y}$ for any point $(x,y)\in E$. This map is clearly rational and is the inverse of $\phi$.

Let's now use this parametrization to solve the question of finding all triples of integer numbers $(a,b,c)$ such that $a^2+b^2=c^2$ (such triples are called Pythagorean triples, since they correspond naturally to right-angles triangles with integer side lengths).

Instead of solving this question we will solve the question of finding all rational solutions of $x^2+y^2=1$ (clearly if $a,b,c$ is a Pythagorean triple, then $\frac{a}{c},\frac{b}{c}$ is a rational solution of $x^2+y^2=1$; to go in the opposite direction one should be a little careful and deal with only primitive Pythagorean triples - those without common factors. But this can also be done quite easily). Now $(x,y)$ is a pair of rational numbers satisfying $x^2+y^2=1$ if and only if $t=\psi(x,y)$ is a rational number! Now we have a complete answer: every rational point $(x,y)$ on teh unit circle can be obtained by applying $\phi$ to a rational number $t$. If we take $t=\frac{p}{q}$ with integer numbers $p,q$ without common factors, then we get the corresponding point $(x,y)=\phi(t)=(\frac{2pq}{p^2+q^2},\frac{p^2-q^2}{p^2+q^2})$. From this parametrization it's not hard to gt that all Pythagorean triples are of the form $(2pqk,k(p^2-q^2),k(p^2+q^2))$ for some integer numbers $p,q,k$.

Another problem we can solve using rational parametrization of conics is the problem of finding integrals of functions of the form $\int \! R(x,\sqrt{ax^2+bx+c}) \, dx$, where $R$ is some rational function of two variables and the expression under the square root is not a complete square. We can look at the conic $E$ given by the equation $y^2=ax^2+bx+c$. Then we have to integrate $R(x,y) \, dx$, where $(x,y)$ is restricted to the conic $E$. To do so we use the rational parametrization of the conic. Let $\phi(t)=(\phi_1(t),\phi_2(t))$ be the rational parametrization by parameter $t$. Make the substitution $t=\psi(x,y)=\psi(x,\sqrt{ax^2+bx+c})$ with the inverse being $x=\phi_1(y)$. Then $dx=\frac{d\phi_1}{dt} dt$ and thus we have to integrate the rational form $R(\phi_1(t),\phi_2(t))\frac{d\phi_1}{dt} dt$ - no square roots are involved now! Once we find this integral we plug in $t=\psi(x,y)=\psi(x,\sqrt{ax^2+bx+c})$ to find the answer to the original question!

Finally in our context we used the central projection from a point on a conic to define cross-ratio of points on a quadric. Essentially the rational parametrization of a quadric makes it indistinguishable (as an abstract, non-embedded curve) from the projective line. Thus one should be able to define every notion that can be defined abstractly for the rational line also for conic sections. We, however, defined the cross-ratio in terms of the embedding, and thus we can't see this point clearly yet.

In courses of algebraic geometry you will study how the ideas we described here evolved to curves of higher degrees, to surfaces and so on.

\section{Homogeneous coordinates}

In this section we will study a way to describe projective geometry and projective transformations which lends itself readily to introducing coordinates and relating geometry to algebra. This introduction of algebra gives us the opportunity to study some geometric questions over different fields, e.g. the field of complex numbers. It also allows us to formulate the general notion of a projective transformation from one projective space to another.

We will start with a rather unusual definition: a projective space on dimension $n$ is the space of lines passing through the origin in a vector space of dimension $n+1$.

Let's think about this definition for a little bit. How does it reflect what we imagine when we talk about projective space?

If we have a hyperplane $\pi$ in the vector space $V$ which doesn't pass through the origin, then each line that does pass through the origin intersects it in a point. Well, almost: the lines that lie in the plane $\pi_\infty$ parallel to $\pi$ and passing through the origin do not intersect $\pi$. Instead of defining a point on $\pi$, such lines define a family of parallel lines inside the plane $\pi$: line $l$ in $\pi_\infty$ defines the family of lines inside $\pi$ which are parallel to it. Thus the space of lines through origin in $V$ is the same as space of points in $\pi$, to which we added a "hyperplane at infinity" - the hyperplane, whose points are families of parallel lines in $\pi$. We see that our new definition is indeed consistent with the way we thought previously about projective space.

Moreover, if we choose a different plane $\pi'$ instead of $\pi$, then to go from one of them to the other all we have to do is to project from point $0\in V$.

We see that our new definition incorporates all our previous intuitions and also is more "self-contained" - it doesn't require the choice of some esoteric "line at infinity". The price we have to pay is the slightly higher level of abstraction - the points in our new definition of a projective space are lines through the origin in a projective space. We will use the following notation below: for every $0\neq x\in V$ the line through the origin containing $x$ will be denoted by $[x]$. The lines $[x]$ and $[y]$ are equal to each other if and only if $x=\lambda y$ for some nonzero $\lambda$.

We can define now what a projective transformation is: if $V,W$ are two vector spaces and $f:V\rightarrow W$ is an invertible linear transformation, then we can define the map $\mathbb{P}f:\mathbb{P}V\rightarrow\mathbb{P}{W}$ by saying that $\mathbb{P}f([x])=[f(x)]$ for any $x\neq 0$. We call transformations that can be obtained in this way projective transformations.

Let us give an example of these notions in the simplest case of a projective line.

By its definition projective line is the space of lines through the origin in a two-dimensional vector space. Choose a basis in this vector space, so that any point in it can be given by a pair of coordinates $(x,y)$. We will denote by $[x:y]$ the line passing through the origin and the point $(x,y)$, provided that $(x,y)\neq (0,0)$. For any nonzero $\lambda$ we have then that $[x:y]=[\lambda x, \lambda y]$. In particular if $y\neq 0$, then $[x:y]=[\frac x y:1]$. And if $y$ is equal to zero, then $[x:y]=[x:0]=[1:0]$. Thus inside our projective line we have a copy of a usual line of points of the form $[z,1]$ and one extra point at infinity $[1:0]$, which we call $\infty$.

Any invertible linear transformation from this space to itself can be written as $f(x,y)=(a x+ b y, c x + d y)$ for some numbers $a,b,c,d$ such that $ad-bc \neq 0$. Let's see how the projective transformation $\mathbb{P}f$ induced from $f$ looks like. The image of a "finite point" $[z:1]$ under $\mathbb{P}f$ is $[a z + b: c z + d]$, or $[\frac{az+b}{cz+d}:1]$, if $cz+d \neq 0$. The point at infinity gets mapped to $[a:c]=[\frac a c:1]$.

In the case our base field is complex numbers we have just recovered the Mobius transformations! The Riemann sphere is now nothing else but the complex projective line $\mathbb{CP}^1$ and the Mobius transformations are the projective transformations from this line to itself!

\section{Projective classification of real quadrics}

If we think of the projective space as the space of lines through the origin in some vector space, then a quadric in this projective space is the space of lines contained in a quadratic cone, i.e. the set of solutions of a homogeneous equation of degree 2.

A degree 2 homogeneous equation can be written as $x\in V: x^T A x=0$ for some symmetric matrix $A$.

Let's see how it is done in $\mathbb{R}^3$: a homogeneous equation of degree 2 is of the form $a x^2 + b x y + c y^2 + d x z + e y z + f z^2=0$. We can write the left-hand side of this equation as $\left(\begin{matrix}x & y & z \end{matrix}\right)\left(\begin{matrix}a & b/2 & d/2 \\ b/2 & c & e/2 \\ d/2 & e/2 & f \end{matrix}\right)\left(\begin{matrix}x \\ y \\ z \end{matrix}\right)=0$.

Now a projective transformation is a transformation coming from a linear transformation of the form $x\to Bx$ for some invertible matrix $B$. The solutions of the equation $x^T A x=0$ get mapped by this transformation to the solutions of the equation $y^T (B^-1)^TAB^-1 y=0$ ($y$ and $x$ are related by $y=Bx$). Thus the study of the classification of quadrics up to projective transformation is the same as the study of symmetric matrices up to transformations of the form $A\to C^T A C$.

But this is a classical question of linear algebra, with a very simple answer to it: by transformation of the form $A \to C^T A C$ a symmetric matrix can be brought in a unique way to a diagonal matrix with $k$ ones along the diagonal, $m$ minus ones and all the rest zeros. The integer $m$ is called the signature of the matrix and the integer $n+m$ is called its rank.

Let's see what this result gives us geometrically for the projective plane. According to the result by a projective transformation every quadric (solution of a quadratic equation) can be brought to one of the following forms (note that we can multiply the equation of a quadric by $-1$ without changing the solution set, so we can always assume that $n\ge m$ above):

Signature 0, rank 3: $x^2+y^2+z^2=0$, the empty quadric (this equation doesn't have any non-zero solutions)

Signature 1, rank 3: $x^2+y^2-z^2=0$, the non-degenerate quadric (in different models of the projective plane can look like a circle, an ellipse, a parabola or a hyperbola)

Signature 0, rank 2: $x^2+y^2=0$, one point (only the point $[0:0:1]$ is the solution of this equation in the projective space)

Signature 1, rank 2: $x^2-y^2=0$, pair of lines $x=\pm y$ (in different models of projective plane they can be parallel or intersecting)

Signature 0, rank 1: $x^2=0$, double line (we see only the line $[0:y:z]$, but it's natural to think of it as a pair of coinciding lines, since it is given by equation of degree two $x^2=0$, rather than the linear equation $x=0$)

Signature 0, rank 0: $0=0$, the whole plane.

In particular this linear algebra shows that in projective geometry there is only one non-degenerate planar quadric, which in different coordinate charts could appear as an ellipse, hyperbola or parabola.

\section{The only conic through five points}

We know that any two points in the Euclidean, affine, or projective plane lie on a line. If the points are distinct, they lie on exactly one line.

In a similar vein, any three non-collinear points lie on a unique circle (in the Euclidean plane - the notion of a circle is not well-defined in affine or projective geometry).

In this section we will try to prove the following theorem:

\begin{theorem} Any five points lie on a quadric. If these points are distinct and no four of them lie on one line, the quadric passing through them is unique. Moreover, the quadric containing the five distinct points is non-degenerate if and only if no three points among these five are collinear.
\end{theorem}

\begin{proof}
Let's start with the simpler part of the statement, which deals with degenerate cases. We first prove that if three distinct points are collinear, then every quadric passing through them is degenerate. Indeed, take the equation of a quadric passing through three collinear points and restrict it to the line containing these points. This equation is quadratic, but it has three roots! (the three points on the line that are assumed to lie on the quadric). Hence this quadratic equation must vanish identically on this line, i.e. the line must be contained in the quadric. Hence this quadric is degenerate (either union of two lines, a double line or the whole plane).

Moreover, in the case three of the five points are collinear, but no four among them are, it's easy to see that the degenerate quadric passing through the five points is unique. Indeed, what we have to do is take the union of the line that contains three collinear points with the line (distinct from the first one) that contains the other two.

Finally if we do have four collinear points among the five, then there are many degenerate quadrics passing through these five points: we can take the union of the line passing through the four collinear points and any line passing through the other one.

Now we show existence of a quadric passing through five given points by a very general argument.

Let $Q$ be any $n$-dimensional space of functions on any space $X$. Let $a_1,\ldots,a_{n-1}$ be any $n-1$ points of this space $X$. Then there exists a nonzero function $f\in Q$ which vanishes on $a_1,\ldots,a_{n-1}$.

For our application we will take $X$ to be $\mathbb{R}^2$, the space $Q$ will be the space of quadratic functions, i.e. functions of the form $a x^2+b xy + cy^2+dx+ey+f$. Since this space is six-dimensional (the parameters $a,b,c,d,e,f$ can be arbitrary, so that the functions $x^2,xy,y^2,x,y,1$ are a basis for this space of functions), the claim above tells us that for every five points $a_1,\ldots,a_5\in \mathbb{R}^2$ there exists a non-zero quadratic function, which vanishes at these five points. The zero set of this function is the quadric passing through the five points $a_1,\ldots,a_5$.

To prove the general case all we have to do is to notice that the condition $f(a_i)=0$ defines a hyperplane in the space of functions $Q$. Indeed, the space of functions vanishing at a point $a_i$ is linear subspace (if $f$ and $g$ both vanish at $a_i$, then all their linear combinations do). Moreover, it is of codimension at most 1 because $f(a_i)=0$ is just one equation on the function $f$ (more precisely, the quotient space of $Q$ by the space $f| f(a_i)=0$ is one-dimensional, since the value of a function at point $a_i$ defines its coset uniquely). If this is somewhat vague, consider the example we are interested in: the condition that a quadratic function $f(x,y)=ax^2+bxy+cy^2+dx+ey+f$ vanishes at point $a_i=(x_i,y_i)$ defines a linear equation on the coefficients $a,b,c,d,e,f$: $x_i^2 a + x_iy_i b + y_i^2 c + x_i d + y_i e + f=0$. In this case it's obvious that it is a hyperplane in the space of coefficients $a,b,c,d,e,f$.

Now if we have $n-1$ linear subspaces of $Q$ of codimension at most 1, their intersection must be of codimension at most $n-1$, i.e. is of dimension at least 1. Hence it must contain at least one non-zero function.

Exercise: prove that there exists a cubic passing through any given collection of 9 points.

What we proved by now is that there exists a quadric passing through any five points in the plane and we investigated the case when at least three of the points are collinear to the end. Now we want to prove that for five points in general position the quadric passing through them is unique.

Suppose to the contrary that there are two quadrics passing through five given points $a_1,a_2,a_3,a_4,a_5$ in general position. Since these points are in general position, both these quadrics must be non-degenerate (if five points lie on a union of two lines, then at least three of them must lie on the same line). Suppose their equations are $Q_1(x)=0$ and $Q_2(x)=0$. First we notice that we can find a whole pencil (i.e. a one-parameter family) of quadrics passing through the five points, namely the family of quadrics of the form $(Q_1+\lambda Q_2)(x)=0$. The crucial claim is that this family must contain a degenerate quadric! Indeed, take the point $a$ of intersection of lines through points $a_1,a_2$ and through points $a_3,a_4$ (if these happen to be parallel then choose the point of intersection of lines through $a_1,a_3$ and $a-2,a_4$). Since $Q_2$ is non-degenerate, it can't pass through any triple of collinear points. In particular $Q_2(a)\neq 0$. So we can take $\lambda=-\frac{Q_1(a)}{Q_2(a)}$ and for this choice of $\lambda$ we have $(Q_1+\lambda Q_2)(a)=0$. The quadric $(Q_1+\lambda Q_2)(a)=0$ is degenerate, since it passes through a triple of collinear points! But this is a contradiction: the degenerate quadric $(Q_1+\lambda Q_2)(x)=0$ passes through five points $a_1,\ldots,a_5$, no three of which are collinear!

\end{proof}

We note that the trick we used in the end of the proof to show that two non-degenerate quadrics can intersect at most at four points can be used to help solving quadratic equations!

Namely, let $ax^4+bx^3+cx^2+dx+e=0$ be a quartic equation we want to solve. Set $y=x^2$. Then the solutions of the quartic equation are the same as the $x$-coordinates of the system of two quadratic equations $Q_1(x,y)=ay^2+bxy+cx^2+dx+e=0$ and $Q_2(x,y)=y-x^2$. Now every solution of these two equations is also a solution of the equation $(Q_1+\lambda Q_2) (x,y)=0$ for any choice of $\lambda$. We want to choose $\lambda$ in such a way that the quadric defined by equation $(Q_1+\lambda Q_2) (x,y)=0$ is degenerate. How can we find such $\lambda$ without finding the intersection points?

It turns out that the quadric $ax^2+bxy+cy^2+dx+ey+f=0$ is degenerate if and only if the determinant of the matrix $\left(\begin{matrix}a & b/2 & d/2 \\ b/2 & c & e/2 \\ d/2 & e/2 & f \end{matrix}\right)$ is equal to zero\footnote{If matrix $A$ has zero determinant, then we can find point $x\neq 0$ with $Ax=0$, and then for every $y$ on the quadric defined by the matrix $A$ (i.e. quadric $y^TAy=0$) the whole line of points $\lambda x+\mu y$ is on the quadric - $(\lambda x+\mu y)^T A (\lambda x+\mu y)=\mu^2 y^TAy=0$, so the quadric is degenerate and $x$ is the vertex of the corresponding cone. Conversely, if the quadric defined by $A$ is degenerate, then it looks like a cone with vertex $x$ for some $x$ and then for every $y$ on the quadric we have $x^T A y=\frac12((x+y)^T A (x+y)-x^T A x -y^T A y)=0$ and hence $A^T x$ is orthogonal to every $y$ on the quadric. Thus $A x=0$.}. If quadrics $Q_1$ and $Q_2$ are defined by matrices $\tilde Q_1$ and $\tilde{Q}_2$, then to find the $\lambda$'s for which the quadric $(Q_1+\lambda Q_2)(x,y)=0$, we have to solve the cubic in the unknown $\lambda$ equation $\det(\tilde{Q}_1+\lambda \tilde Q_2)=0$.

So to solve a quartic equation we have to intersect two quadrics $Q_1(x,y)=0$ and $Q_2(x,y)=0$. Instead we can solve a cubic equation $\det(\tilde Q_1+\lambda \tilde Q_2)=0$ to find $\lambda$ for which the quadric $(Q_1+\lambda Q_2)(x,y)=0$ becomes a pair of two lines and then intersect the quadric $Q_1(x,y)=0$ with this pair of two lines. To do so we have to solve only a pair of quadratic equations!