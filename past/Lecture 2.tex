\chapter{The Technique of Center of Mass}
\section{Intuitive meaning of center of mass}
The notion of center of mass of a body or a system of particles, which is very useful in kinematics, proves to be very handy in geometry as well.
Let's see what the center of mass is for a system of two point masses $m_1$ and $m_2$ located at points $A$ and $B$. One way to define it is to say it is the unique point $C$ on the line $AB$ such that $m_1\cdot AC=m_2\cdot CB$ (the notations $AC$, $CB$ can be interpreted as just lengths of the segments for now, but later, when we introduce negative masses, it will be better to think of them as oriented lenths, or, even better, vectors).
\begin{figure}[h]
\centering
\begin{asy}
size (3cm);
pair A, B, C;
A=(0,0);
B=(4.5,3);
C=(3,2);
draw(A--B);
dot("$A$",A,SW);
label("$m_1$",A,NW);
dot("$B$",B,SE);
label("$m_2$",B,NE);
dot("$C$",C,N);
\end{asy}
\caption{If the mass at $B$ is twice as big as the mass at $A$, then the center of mass $C$ will be twice as close to $B$ as it is to $A$}
\label{fig:center_of_mass_def}
\end{figure}Type in the content of your page here.
One possible way to understand what center of mass means is to imagine a big flat weightless table top with two small objects of masses $m_1$ and $m_2$ pulling down (with forces proportional to their masses). If we want to support the table top on just one leg, we will have to place this leg at the center of mass of these two objects.
\begin{figure}[h]
\centering
\begin{asy}
size (4cm);
pair A, B, C;
A=(3,4);
B=(6,5.5);
C=(5,5);
draw((0,3)--(2,7)--(10,7)--(8,3)--cycle);
draw((4.8,0)--(5,5)--(5.2,0)--cycle);
draw(A--B);
dot("$A$",A,NW);
dot("$B$",B,NE);
dot("$C$",C,N);
draw(A--(A-(0,2)),EndArrow);
label("$m_1$",A-(0,2),SW);
draw(B--(B-(0,1)),EndArrow);
label("$m_2$",B-(0,1),SE);
draw(C--(C+(0,3)),EndArrow);
label("$m_1+m_2$",C+(0,3),N);
\end{asy}
\caption{Force proportional to $m_1+m_2$ must be placed at $C$ to keep the table in equilibrium}
\label{fig:center_of_mass_intuition}
\end{figure}
This is also the intuitive meaning of the center of mass of any collection of objects - it is the point at which a weightless tabletop should be supported if it has to stay in equilibrium once we put our system of objects on it.
For the sake of equilibrium considerations we can think of the effect of putting point masses $m_1$ and $m_2$ at points $A$ and $B$ as equal to the effect of placing mass $m_1+m_2$ at their center of mass (see figure \ref{fig:center_of_mass_intuition}).
We will use this property to define the center of mass of three point masses $m_1,m_2$ and $m_3$ placed at points $A_1,A_2$ and $A_3$ respectively. We can first replace the two masses $m_1$ and $m_2$ by one mass $m_1+m_2$ and place this new mass at the center of masses $C_{12}$ of the masses $m_1$ and $m_2$. As we mentioned above, this shouldn't change the equilibrium of the tabletop, thus the center of mass of the system shouldn't change as well. So to compute the center of mass of all three bodies it remains to find the center of mass of mass $m_3$ at $A_3$ and the combined mass $m_1+m_2$ at $C_{12}$.
Note, however, that there are several problems with the process of finding the center of mass of three bodies we just described. First of all, we did not define rigorously what we mean by center of mass. The tabletop equilibrium considerations depend on our intuitive understanding of the words "mass", "flat weightless tabletop", "forces" etc. One might argue that we can \textit{define} \ the center of masses as the answer that the process of finding it described above produces. While this is what we shall do presently, such definition is also problematic. Indeed, why did we combine the masses in the following order: first combine $m_1$ with $m_2$ and then combine the result with $m_3$? Why not combine $m_2$ with $m_3$ first and then combine the result with $m_1$? Maybe we will get a different answer and then our "definition" won't be applicable? We will see in next couple of sections that this is not the case. In fact the property that all the computations of the center of mass by using different processes (i.e. different sequences of combining masses into groups) give the same answer will be very useful in geometric problems.
\section{Axiomatic definition of center of mass}
To summarize the discussion above, we give now a list of axioms center of mass should satisfy. The fact that there is a construction that associates to a system of masses its center of mass that satisfies these axioms will be proved later, in section ...
Axiom 1: the center of mass of a system of two point masses $m_1$ and $m_2$ located at points $A$, $B$ is the unique point $C$ on the line $AB$ such that $m_1\cdot AC=m_2\cdot CB$ (the masses $m_1$, $m_2$ are allowed to be any real numbers).
Axiom 2: Suppose the system of masses $S$ is subdivided to disjoint subsystems $S_1,\ldots,S_k$. Suppose that the total mass of subsystem $S_i$ is $M_i$ and that its center of masses is located at point $C_i$. Then the center of mass of system $S$ coincides with the center of mass of system of masses $M_i$ located at points $C_i$.
Using the two axioms above, we can compute the center of mass of any system of point masses: indeed, axiom 1 allows one to compute the center of mass of any system of two masses, and axiom 2 allows recursive calculation of C.M. of any larger system of masses by subdividing it to several subsystems.
The axioms above do not, however, justify why they are consistent, i.e. why there is a way to assign center of mass to any system of masses in a way that both axioms will be satisfied. We will deal with this in a later section. Now we will assume that the center of masses is well-defined and finally look at some examples when this notion can be applied to problems from geometry.
\section{Applications of notion of Center of Mass}
Suppose we start with a triangle $ABC$ and three unit masses at its vertices. How can we compute the center of mass of the resulting system? One way is to combine masses at $A$ and $B$ first. This will give us a mass of two units placed at the midpoint of $AB$. Now we combine the resulting mass with the unit mass at $C$. By doing so we get that the center of mass lies on the segment connecting the vertex $C$ to the midpoint of $AB$ and divides it in ratio $2:1$ (the part of it adjacent to the vertex being the longer one). But we could proceed in a different way: first combine the masses at $A$ and $C$ and then combine the result with the mass at $B$. In this way we get that the center of mass of the system lies on the median from vertex $B$ and divides it in ratio $2:1$. By combining the masses in the third possible order ($B$ and $C$ first and adding $A$ afterwards) we get that it also lies on the median from the vertex $A$ and divides it in the ratio $2:1$. Thus we get that the center of mass lies on all three of the median of triangle $ABC$ and divides them in ratios $2:1$. We've seen this theorem in previous lecture, but didn't interpret the intersection point of medians as the center of mass.
The argument above is very simple, but it is also very typical of how the notion of center of mass is used in geometry. We will see in a moment more complicated applications, but they all use the same simple ideas.
For slightly more sophisticated example of use of notion of center of mass, we can reprove Ceva's theorem. Let's remember what it says:
\begin{theorem}
Let $ABC$ be a triangle and $A', B', C'$ be points on the sides $BC, CA$ and $AB$ respectively. Then the three lines $AA', BB', CC'$ are concurrent if and only if $\frac{BA'}{A'C}\cdot\frac{CB'}{B'A}\cdot\frac{AC'}{C'B}=1$.
\end{theorem}
\begin{proof}
Place masses $m_A$, $m_B$ and $m_C$ at points $A,B,C$ so that $m_A\cdot AC' = m_B \cdot C'B$ and $m_B\cdot BA'=m_C A'C$ (note that we assume the points $A',B',C'$ are given to us and we are choosing the masses). For instance we can choose $m_A=1$, $m_B=\frac{AC'}{C'B}$ and $m_C=\frac{AC'}{C'B}\cdot\frac{BA'}{A'C}$. Then the center of mass of $m_A$ and $m_B$ is at point $C'$ and the center of mass of masses $m_B,m_C$ is at point $A'$. Thus the center of masses of all three must lie both on the line $CC'$ (if we combine the masses at $A,B$ first and then combine the result with the mass at $C$) and on the line $AA'$ (if instead we combine the masses at $B$ and $C$ first). So the center of masses lies on the intersection of the lines $AA'$ and $CC'$. We will call it $O$. Our goal now is to show that $O$ lies on the line $BB'$ if and only if $\frac{BA'}{A'C}\cdot\frac{CB'}{B'A}\cdot\frac{AC'}{C'B}=1$.
Now $O$, being the center of mass, lies on the segment joining $B$ with the center of mass of $m_A$ and $m_C$. The latter divides the segment $AC$ in ratio $\frac{m_C}{m_A}=\frac{AC'}{C'B}\cdot\frac{BA'}{A'C}$.
Hence $B'$ is this center of mass (that is $O$ belongs to $BB'$) if and only if $\frac{m_C}{m_A}$ is equal to $\frac{AB'}{B'C}$, or equivallently $\frac{AC'}{C'B}\cdot\frac{BA'}{A'C}=\frac{B'A}{CB'}$. The last condition is the same as $\frac{AC'}{C'B}\cdot\frac{BA'}{A'C}\cdot\frac{CB'}{B'A}=1$, as required.
\end{proof}
Note how a clever choice of masses in the last proof allowed us to prove both directions of Ceva's theorem in a rather simple fashion.
A similar technique can be adopted to prove Menelaus' theorem as well.
\begin{theorem}
Let $ABC$ be a triangle and let points $A_1,B_1$ and $C_1$ belong to the lines $BC, CA$ and $AB$ respectively.
The points $A', B'$ and $C'$ belong to one line if and only if $\frac{BA'}{A'C}\cdot\frac{CB'}{B'A}\cdot\frac{AC'}{C'B}=-1$
\end{theorem}
\begin{proof}
Place masses $m_A,m_B$ and $m_{B'}$ at points $A,B$ and $B'$ respectively in such a way that $C'$ is the center of mass of $m_A$ and $m_B$ (i.e. $\frac{m_A}{m_B}=\frac{C'B}{AC'}$) while the center of mass of $m_A$ and $m_{B'}$ is at $C$ (i.e. $\frac{m_B'}{m_A}=\frac{AC}{CB'}$).
The center of mass of all three then lies both on the segment $B'C'$ and on the segment $BC$. Call this point $O$.
We want to show that $O$ coincides with $A'$ if and only if $\frac{BA'}{A'C}\cdot\frac{CB'}{B'A}\cdot\frac{AC'}{C'B}=-1$.
Now point $O$ divides the segment $BC$ in ratio $\frac{BO}{OC}=\frac{m_A+m_{B'}}{m_B}$, hence it coincides with point $A'$ if and only if $\frac{BA'}{A'C}=\frac{m_A+m_{B'}}{m_B}$. But the ratio $\frac{m_A+m_{B'}}{m_B}$ can be computed as follows: $$\frac{m_A+m_{B'}}{m_B}=\frac{m_A+m_{B'}}{m_A}\cdot \frac{m_A}{m_B}$$
The ratio $\frac{m_A+m_{B'}}{m_A}=1+\frac{m_{B'}}{m_A}$ is equal to $1+\frac{AC}{CB'}=\frac{AC+CB'}{CB'}=\frac{AB'}{CB'}$, while the ratio $\frac{m_A}{m_B}$ is equal to $\frac{C'B}{AC'}$. Thus we can conclude that the point $O$ coincides with $A'$ if and only if $\frac{BA'}{A'C}=\frac{AB'}{CB'}\cdot\frac{C'B}{AC'}$ or $\frac{AB'}{CB'}\cdot\frac{C'B}{AC'}\cdot\frac{A'C}{BA'}=1$. The last equality can be rewritten in the form $\frac{AB'}{B'C}\cdot\frac{BC'}{C'A}\cdot\frac{CA'}{A'B}=-1$, as required.
\end{proof}
\section{Analogues of theorems of Ceva and Menelaus in space}
After having seen how the notion of center of mass helps to prove Ceva's and Menelaus' theorems, we can try to find and prove three-dimensional analogues of these theorems. In fact instead of having two separate theorems, like in plane, we have only one Ceva-Menelaus theorem, but it is simultaneously the analogue of both planar theorems. Here is what it says:
\begin{theorem}
Let $ABCD$ be a non-planar quadrilateral in space. Let points $X,Y,Z,W$ lie on lines $AB,BC,CD$ and $DA$ respectively. Then the following three conditions are equivallent:
1. The four points $X,Y,Z,W$ are coplanar.
2. The four planes $ABZ,BCW,CDX,DAY$ are concurrent.
3. $\frac{AX}{XB}\cdot\frac{BY}{YC}\cdot\frac{CZ}{ZD}\cdot\frac{DW}{WA}=1$
\end{theorem}
\input{./CenterOfMass/figCevaMenelaus3D.tex}
\input{./CenterOfMass/figCevaMenelaus3D2.tex}
Notice that it is reasonable that the same condition $\frac{AX}{XB}\cdot\frac{BY}{YC}\cdot\frac{CZ}{ZD}\cdot\frac{DW}{WA}=1$ appear in both Ceva's and Menelaus' theorem in space, because the product $\frac{AX}{XB}\cdot\frac{BY}{YC}\cdot\frac{CZ}{ZD}\cdot\frac{DW}{WA}$ is equal to $(-1)^4$ times the product $\frac{XA}{BX}\cdot\frac{YB}{CY}\cdot\frac{ZC}{DZ}\cdot\frac{WD}{AW}$, so that they are in fact equal.
\begin{proof}
Condition 1 is equivalent to the condition that lines $XZ$ and $WY$ intersect at a common point (indeed, two lines in space intersect if and only if there is a plane containing both of them). Now the line $XZ$ is the intersection line of planes $ABZ$ and $DCX$ while the line $WY$ is the intersection line of planes $BCW$ and $ABZ$. Thus the lines $XZ$ and $WY$ intersect if and only if the four planes are concurrent.
Now we will present two proofs that condition 1 is equivalent to condition 3. The first proof relies on the notion of center of mass.
Place masses $m_A,m_B,m_C$ and $m_D$ at points $A,B,C,D$ so that the center of mass of $m_A$ and $m_B$ is at $X$, the center of mass of $m_B$ and $m_C$ is at $Y$ and finally the center of mass of $m_C$ and $m_D$ is at $Z$. These conditions are equivalent to $\frac{m_B}{m_A}=\frac{AX}{XB},\frac{m_C}{m_B}=\frac{BY}{YC}$ and $\frac{m_D}{m_C}=\frac{CZ}{ZD}$ and they can be easily satisfied. Let $\tilde{W}$ be the center of mass of $m_A$ and $m_D$.
If we combine the masses at $A$ and $B$ and also combine the masses at $C$ and $D$, we get that the total center of mass lies on the line $XZ$. If we combine the masses at $A$ and $D$ and also combine the masses at $B$ and $C$, we get that the same point lies on $Y\tilde{W}$.
Thus the lines $XZ$ and $Y\tilde{W}$ intersect at one point, hence the four points $X,Y,Z,\tilde{W}$ are coplanar.
It follows that the points $X,Y,Z,W$ can be coplanar if and only if $\tilde{W}$, the center of mass of $m_A$ and $m_D$, coincides with $W$, or, equivalently, $\frac{m_D}{m_A}=\frac{AW}{WD}$. But $\frac{m_D}{m_A}=\frac{m_D}{m_C}\cdot\frac{m_C}{m_B}\cdot\frac{m_B}{m_A}=\frac{AX}{XB}\cdot\frac{BY}{YC}\cdot\frac{CZ}{ZD}$. So $X,Y,Z,W$ are coplanar if and only if $\frac{AX}{XB}\cdot\frac{BY}{YC}\cdot\frac{CZ}{ZD}=\frac{AW}{WD}$ or $\frac{AX}{XB}\cdot\frac{BY}{YC}\cdot\frac{CZ}{ZD}\cdot\frac{DW}{WA}=1$, as required.
Another way to prove the same result is along the lines of proof of Menelaus' theorem in lecture 1. We first prove a lemma about parallel projection on a line.
\input{./CenterOfMass/figParallelProjectionInSpace.tex}
\begin{lemma}
Let $l,l'$ be two lines. Let $\pi_1,\pi_2,\pi_3$ be three plnes that are parallel to each other and intersect both $l$ and $l'$. Let $A_1,A_2,A_3$ be the intersection points of $\pi_1,\pi_2$ and $\pi_3$ with the line $l$. Let also $A_1',A_2',A_3'$ be the intersection points of $\pi_1,\pi_2$ and $\pi_3$ with the line $l'$. Then $\frac{A_1A_2}{A_2A_3}=\frac{A_1'A_2'}{A_2'A_3'}$.
\end{lemma}
This lemma states that parallel projection from a line to another line in space preserves ratios of oriented lengths.
\begin{proof}
In case line $l$ and $l'$ lie in a common plane, the claim follows from the two-dimensional analogue (the parallel projection in the space restricts to a parallel projection in two dimensions).
\input{./CenterOfMass/figParallelProjectionInSpace1.tex}
In the case when the lines $l,l'$ do not lie in a common plane, we can find another line $l''$ that intersects both of these lines and is not parallel to the planes $\pi_1,\pi_2,\pi_3$. Let $A_1'',A_2'',A_3''$ denote the points of intersection of line $l''$ with planes $\pi_1,\pi_2,\pi_3$. Since the lines $l,l''$ are coplanar, we can conclude from the first case we considered that $\frac{A_1A_2}{A_2A_3}=\frac{A_1''A_2''}{A_2''A_3''}$. Similarly the lines $l''$ and $l'$ are coplanar, hence $\frac{A_1''A_2''}{A_2''A_3''}=\frac{A_1'A_2'}{A_2'A_3'}$. These two equalities together imply that $\frac{A_1A_2}{A_2A_3}=\frac{A_1'A_2'}{A_2'A_3'}$.
\input{./CenterOfMass/figParallelProjectionInSpace2.tex}
\end{proof}
With the help of this lemma we can now finish the proof of Ceva-Menelaus theorem. Suppose that the points $X,Y,Z,W$ lie in one plane. Choose a line $l$ that is not parallel to this plane and consider parallel projection onto this line along the direction of the plane $XYZW$. Let $A',B',C',D'$ be the images of points $A,B,C,D$ under this projection. Let $O$ be the image of points $X,Y,Z,W$. Then clearly $\frac{A'O}{OB'}\cdot\frac{B'O}{OC'}\cdot\frac{C'O}{OD'}\cdot\frac{D'O}{OA'}=1$ (because of cancellations) and the lemma implies that $\frac{A'O}{OB'}=\frac{AX}{XB}$,$\frac{B'O}{OC'}=\frac{BY}{YC}$,$\frac{C'O}{OD'}=\frac{CZ}{ZD}$ and $\frac{D'O}{OA'}=\frac{DW}{WA}$. Thus $\frac{AX}{XB}\cdot\frac{BY}{YC}\cdot\frac{CZ}{ZD}\cdot\frac{DW}{WA}=1$.
Conversely, suppose that $\frac{AX}{XB}\cdot\frac{BY}{YC}\cdot\frac{CZ}{ZD}\cdot\frac{DW}{WA}=1$. Let $\tilde{W}$ be the point of intersection of line $AD$ with the plane $XYZ$. From what we already proved, $\frac{AX}{XB}\cdot\frac{BY}{YC}\cdot\frac{CZ}{ZD}\cdot\frac{D\tilde{W}}{\tilde{W}A}=1$. Hence $\frac{D\tilde{W}}{\tilde{W}A}=\frac{DW}{WA}$, showing that $\tilde{W}=W$.
\end{proof}
\section{Existence of Center of Mass}
After having seen some of the applications of the notion of center of mass, time has come to make sure one exists, i.e. that there is a rule that associates a point to a system of masses, which satisfies axioms 1 and 2 from section ...
We will find the center of mass of a system of particles by introducing linear structure to our space, i.e. choosing an origin allowing ourselves to add and scale points.
Recall that if we choose an origin $O$ in the plane, then we can identify any point $A$ in the plane with the vector $\vect{OA}$ from the origin $O$ to $A$. Once we do so, we get the ability to speak about sum of points and scalar multiples of points. Note however that the results of scaling and addition depend on the particular choice of origin. Indeed, if we choose another point $O'$ as the origin, then the point $A$ will correspond to the vector $\vect{O'A}$. This vector can be expressed as $\vect{O'O}+\vect{OA}$. Similarly point $B$ corresponds now to vector $\vect{O'B}=\vect{O'O}+\vect{OB}$. The point, expressing the sum of $\vect{OA}+\vect{OB}$ in previous coordinate system corresponds now to the vector $\vect{O'O}+(\vect{OA}+\vect{OB})$. However the sum of $\vect{O'A}$ and $\vect{O'B}$ is equal to $(\vect{O'O}+\vect{OA})+(\vect{O'O}+\vect{OB})=2\vect{O'O}+(\vect{OA}+\vect{OB})$, so it doesn't correspond to the same point.
\input{./CenterOfMass/figVectorSum.tex}
The considerations above show that it doesn't make sense to talk about "sum of two points in the plane". This sum can be given some meaning only if an origin is chosen and the meaning depend on the choice. We may ask ourselves whether at least some linear combinations make sense. For example we know that the sum $\frac{1}{2}(\vect{OA}+\vect{OB})$ expresses the midpoint of the segment $AB$, so it has geometrical meaning, which is independent of the choice of origin.
Indeed, if $O'$ is another origin, then the new sum $\frac{1}{2}(\vect{OA}+\vect{OB})$ is equal to $\frac{1}{2}((\vect{O
O}+\vect{OA})+(\vect{O'O}\vect{OB}))=\vect{O'O}+\frac{1}{2}(\vect{OA}+\vect{OB})$, which corresponds in the new coordinates to the same point as $\frac{1}{2}(\vect{OA}+\vect{OB})$ corresponds to in the old one.
We can generalize the previous observation to the following claim.
\begin{claim}
Let $A_i$, $i$ from 1 to $n$, be $n$ points (in plane or in higher dimensional space). Let $\lambda_i$ be real numbers such that thir sum is equal to 1: $\sum_{i=1}^{n}{\lambda_i}=1$. Then the point that corresponds to the vector sum $\sum_1^n{\lambda_i\vect{OA_i}}$ does not depend on the choice of origin $O$.
\end{claim}
\begin{proof}
Let $O'$ be another origin. The the points $A_i$ correspond in the new vector space to vectors $\vect{O'A_i}=\vect{O'O}+\vect{OA_i}$. The vector sum we consider then becomes $\sum_1^n{\lambda_i\vect{O'A_i}}=\sum{\lambda_i(\vect{O'O}+\vect{OA_i})}=\sum{\lambda_i \vect{O'O}}+\sum{\lambda_i\vect{OA_i}}=\vect{O'O}+\sum{\lambda_i\vect{OA_i}}$. The point of our space that corresponds to the vector $\vect{O'O}+\sum{\lambda_i\vect{OA_i}}$ in the new coordinate system is the same as the point to which $\sum{\lambda_i\vect{OA_i}}$ corresponds in the old one.
\end{proof}
Since the point that corresponds to the sum $\sum_1^n{\lambda_i\vect{OA_i}}$ doesn't depend on the choice of origin (provided that $\sum_{i=1}^{n}{\lambda_i}=1$), we will omit the point $O$ from the notation and will write the point that corresponds to the sum $\sum_1^n{\lambda_i\vect{OA_i}}$ simply as $\sum_1^n{\lambda_i A_i}$.
Now we will define center of mass of a system of point masses $m_i$ located at points $A_i$ as the vector sum $\frac{\sum_i{m_i A_i}}{M}$, where $M=\sum_i{m_i}$ is the total mass of the system. We are allowed to do this, because $\sum_i{\frac{m_i}{M}}=1$.
It remains to verify that the axioms 1 and 2 from section ... are satisfied by this defintion.
For axiom 1, let $m_1,m_2$ be two masses located at points $A$ and $B$. Let $C=\frac{m_1 A+m_2 B}{m_1+m_2}$ be their center of mass. We should verify that $m_1 \vect{AC}=m_2 \vect{CB}$. Indeed, since the point $C$ corresponds to vector $\vect{OC}=\frac{m_1 \vect{OA}+m_2 \vect{OB}}{m_1+m_2}$, we have $$m_1 \vect{AC}=m_1(\vect{OC}-\vect{OA})=m_1(\frac{m_1 \vect{OA}+m_2 \vect{OB}}{m_1+m_2}-\vect{OA})=\frac{m_1 m_2(\vect{OB}-\vect{OA})}{m_1+m_2}=\frac{m_1 m_2 \vect{AB}}{m_1+m_2}$$.
Similarly $m_2 \vect{CB}=m_2(\vect{OB}-\vect{OC})=m_2(\vect{OB}-\frac{m_1 \vect{OA}+m_2\vect{OB}}{m_1+m_2})=\frac{m_1m_2(\vect{OB}-\vect{OA})}{m_1+m_2}=\frac{m_1m_2\vect{AB}}{m_1+m_2}$, so indeed $m_1\vect{AC}=m_2\vect{CB}$.
Now for verification of axiom 2 we will need more or less the same amount of algebra, but slightly more sophisticated notation. Let $S$ be a system of point masses subdivided to subsystems $S_1,\ldots,S_n$. For a point mass $x$ we will denote by $m_x$ its mass and by $A_x$ - the point at which it is located.
The center of mass of the system $S$ corresponds (after a choice of origin) to the vector $$\frac{\sum_{x \textrm{in} S}{m_x \vect{OA_x}}}{\sum_{x \textrm{in} S}{m_x}}$$
Similarly the center of mass of system $S_i$ is located at point $C_i$ that corresponds to vector $$\vect{OC_i}=\frac{\sum_{x \textrm{in} S_i}{m_x \vect{OA_x}}}{M_i}$$, where $M_i=\sum_{x \textrm{in} S_i}{m_x}$ is the total mass of the subsystem $S_i$.
We want to show that $$\frac{\sum_{x \textrm{in} S}{m_x \vect{OA_x}}}{\sum_{x \textrm{in} S}{m_x}}=\frac{\sum_i{M_i\vect{OC_i}}}{\sum_i{M_i}}$$
Notice that $\sum_i{M_i}=\sum_i{\sum_{x \textrm{in} S_i}{m_x}}=\sum_{x \textrm{in} S}{m_x}$ (the total mass of the system is the sum of the masses of all the subsystems), so it remains to show that $\sum_{x \textrm{in} S}{m_x \vect{OA_x}}=\sum_i{M_i\vect{OC_i}}$. Remember that $$\vect{OC_i}=\frac{\sum_{x \textrm{in} S_i}{m_x \vect{OA_x}}}{M_i}$$ so we have to prove that $\sum_{x \textrm{in} S}{m_x \vect{OA_x}}=\sum_i{\sum_{x \textrm{in} S_i}{m_x \vect{OA_x}}}$, which is obvious.
Now we've finally established the existence of center of mass and some of its its interesting properties.
Add exercise on moment of inertia.